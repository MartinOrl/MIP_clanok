% Metódy inžinierskej práce

\documentclass[8pt,twoside,slovak,a4paper]{article}

\usepackage[slovak]{babel}
%\usepackage[T1]{fontenc}
\usepackage[IL2]{fontenc} % lepšia sadzba písmena Ľ než v T1
\usepackage[utf8]{inputenc}
\usepackage{graphicx}
\usepackage{url} % príkaz \url na formátovanie URL
\usepackage{hyperref} % odkazy v texte budú aktívne (pri niektorých triedach dokumentov spôsobuje posun textu)

\usepackage{cite}
%\usepackage{times}

\pagestyle{headings}

\title{Metódy strojového učenia a ich praktické použitie\thanks{Semestrálny projekt v predmete Metódy inžinierskej práce, ak. rok 2021/22, vedenie: Ing. Fedor Lehocki}} 

\author{Martin Orlej\\[2pt]
	{\small Slovenská technická univerzita v Bratislave}\\
	{\small Fakulta informatiky a informačných technológií}\\
	}

\date{\small 11. oktober 2021} 



\begin{document}

\maketitle

\begin{abstract}
Strojové učenie sa v posledných rokoch čoraz viac spomína či už vo vedeckých prácach, alebo v rôznych článkoch, zameraných na technológie. Či už ide o niečo jednoduché, ako aplikácie na telefóny, alebo o vysoko pokročilé technológie ako autonómne jazdenie a počítačové videnie. V mojej práci by som sa chcel zamerať na rôzne modely strojového učenia, ako napr. lineárna regresia, Boltzmannove stroje alebo transformátory, na ich praktické a najefektívnejšie využitie, napríklad pri spracovávaní veľkého množstva údajov, spracovávaní jazyka (NLP) alebo počítačovom videní a klasifikácii, a na slabé miesta a nevýhody týchto modelov. Rád by som taktiež jednoducho opísal aj už existujúci systém, ako algoritmus GPT-3 vyvinutý nadáciou OpenAI, a možnosti jeho využitia.
\end{abstract}






\section{Úvod} \label {uvod}
V tejto časti je úvod

\subsection{Motivácia} \label{motivacia}

\section{Regresia} \label{regresia}
\subsection{Lineárna regresia} \label{linearnareg}
\subsection{Decision tree}
\subsection{Random Forest}
\subsection{Neurónová sieť}

\section{Klasifikácia} \label{regresia}
\subsection{Logická regresia} \label{logickareg}
\subsection{Support Vector Machine}
\subsection{Naive Bayes}
\subsection{Neurónová sieť}

\section{Praktické využitie strojového učenia} \label{vyuzitie}
\subsection{Medicína}
\subsection{Spracovávanie dát}
\subsection{Poľnohospodárstvo}
\subsection{Automatizácia}
\subsection{Autopilot}

\section{Model GPT-3} \label{gpt}

\section{Diskusia} \label{diskusia}

\section{Záver} \label{zaver} % prípadne iný variant názvu



%\acknowledgement{Ak niekomu chcete poďakovať\ldots}


% týmto sa generuje zoznam literatúry z obsahu súboru literatura.bib podľa toho, na čo sa v článku odkazujete
%\bibliography{literatura}
%\bibliographystyle{plain} % prípadne alpha, abbrv alebo hociktorý iný
\end{document}
